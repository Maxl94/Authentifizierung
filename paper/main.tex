\documentclass{sigchi}

%% EXAMPLE BEGIN -- HOW TO OVERRIDE THE DEFAULT COPYRIGHT STRIP -- (July 22, 2013 - Paul Baumann)
\toappear{}
%% EXAMPLE END -- HOW TO OVERRIDE THE DEFAULT COPYRIGHT STRIP -- (July 22, 2013 - Paul Baumann)

\pagenumbering{arabic}

% Load basic packages
\usepackage{balance}  % to better equalize the last page
\usepackage{graphics} % for EPS, load graphicx instead 
\usepackage[T1]{fontenc}
\usepackage{txfonts}
\usepackage{mathptmx}
\usepackage[pdftex]{hyperref}
\usepackage{color}
\usepackage{booktabs}
\usepackage{textcomp}
% Some optional stuff you might like/need.
\usepackage{microtype} % Improved Tracking and Kerning
% \usepackage[all]{hypcap}  % Fixes bug in hyperref caption linking
\usepackage{ccicons}  % Cite your images correctly!
\usepackage[utf8]{inputenc} % for a UTF8 editor only
\usepackage[ngerman]{babel}

% If you want to use todo notes, marginpars etc. during creation of your draft document, you
% have to enable the "chi_draft" option for the document class. To do this, change the very first
% line to: "\documentclass[chi_draft]{sigchi}". You can then place todo notes by using the "\todo{...}"
% command. Make sure to disable the draft option again before submitting your final document.
\usepackage{todonotes}

% Paper metadata (use plain text, for PDF inclusion and later
% re-using, if desired).  Use \emtpyauthor when submitting for review
% so you remain anonymous.
\def\plaintitle{Lock-a-dos: Ein Rucksack mit Selbstschutz}
\def\plainauthor{Alexander Schuhmann, Max ??????, Michael Stadler, Maximilian Pachl}
\def\emptyauthor{}
\def\plainkeywords{Rucksack; Diebstahl; Selbstüberwachung; Smartphone; Sensoren; Wearable}
\def\plaingeneralterms{Dokumentation}

% llt: Define a global style for URLs, rather that the default one
\makeatletter
\def\url@leostyle{%
  \@ifundefined{selectfont}{
    \def\UrlFont{\sf}
  }{
    \def\UrlFont{\small\bf\ttfamily}
  }}
\makeatother
\urlstyle{leo}

% To make various LaTeX processors do the right thing with page size.
\def\pprw{8.5in}
\def\pprh{11in}
\special{papersize=\pprw,\pprh}
\setlength{\paperwidth}{\pprw}
\setlength{\paperheight}{\pprh}
\setlength{\pdfpagewidth}{\pprw}
\setlength{\pdfpageheight}{\pprh}

% Make sure hyperref comes last of your loaded packages, to give it a
% fighting chance of not being over-written, since its job is to
% redefine many LaTeX commands.
\definecolor{linkColor}{RGB}{6,125,233}
\hypersetup{%
  pdftitle={\plaintitle},
% Use \plainauthor for final version.
%  pdfauthor={\plainauthor},
  pdfauthor={\emptyauthor},
  pdfkeywords={\plainkeywords},
  bookmarksnumbered,
  pdfstartview={FitH},
  colorlinks,
  citecolor=black,
  filecolor=black,
  linkcolor=black,
  urlcolor=linkColor,
  breaklinks=true,
}

% create a shortcut to typeset table headings
% \newcommand\tabhead[1]{\small\textbf{#1}}

% End of preamble. Here it comes the document.
\begin{document}

\title{\plaintitle}

\numberofauthors{4}
\author{%
  \alignauthor{Alexander Schuhmann\\
    \email{e-mail address}}\\
  \alignauthor{Max ??????\\
    \email{e-mail address}}\\
  \alignauthor{Michael Stadler\\
    \email{e-mail address}}\\
  \alignauthor{Maximilian Pachl\\
    \email{pachl@hm.edu}}\\
}

\maketitle

\begin{abstract}
  Aktualisiert---\today. Dieses Paper beschreibt die prototypische
  Umsetzung eines Rucksacks, der sich selbst vor dem Zugriff Dritter
  schützen kann. Dazu werden eine Reihe von Sensoren in den Rucksack
  integriert um mögliche Einflüsse von Außen jederzeit erkennen zu
  können. Ein Alarm sorgt dafür, dass der unbefugte Zugriff nicht
  unbemerkt bleibt. Mithilfe eines Smartphones kann der Besitzer
  die Selbstüberwachung des Rucksackes ein- und ausschalten.
\end{abstract}

\subsection{Schlagwörter}
\plainkeywords

\section{Einleitung}

This format is to be used for submissions that are published in the
conference proceedings. We wish to give this volume a consistent,
high-quality appearance. We therefore ask that authors follow some
simple guidelines. You should format your paper exactly like this
document. The easiest way to do this is to replace the content with
your own material.  This document describes how to prepare your
submissions using \LaTeX.

\section{Fazit}

It is important that you write for the SIGCHI audience. Please read
previous years' proceedings to understand the writing style and
conventions that successful authors have used. It is particularly
important that you state clearly what you have done, not merely what
you plan to do, and explain how your work is different from previously
published work, i.e., the unique contribution that your work makes to
the field. Please consider what the reader will learn from your
submission, and how they will find your work useful. If you write with
these questions in mind, your work is more likely to be successful,
both in being accepted into the conference, and in influencing the
work of our field.

\balance{}

% REFERENCES FORMAT
% References must be the same font size as other body text.
\bibliographystyle{SIGCHI-Reference-Format}
\bibliography{main}

\end{document}
